\documentclass[a4paper]{report}

\usepackage[margin=3.0cm]{geometry}
\usepackage{amsmath}
\usepackage[pdftex]{graphicx}
%\usepackage{graphics}
\usepackage{subfig}



\title{BLASFEO reference guide}
\author{Gianluca Frison}



\begin{document}

\maketitle
\tableofcontents





%%%%%%%%%%%%%%%%%%%%%%%%%%%%%%%%
\chapter{Introduction}
%%%%%%%%%%%%%%%%%%%%%%%%%%%%%%%%

BLASFEO - BLAS For Embedded Optimization.





%%%%%%%%%%%%%%%%%%%%%%%%%%%%%%%%
\chapter{Matrix data type}
%%%%%%%%%%%%%%%%%%%%%%%%%%%%%%%%

The fundamental data type in BLASFEO is a C struct defining a matrix, called {\tt strmat}.
Depending on the chosen linear algebra library, the struct is defined differently.



\section{{\tt strmat} definition}


\subsection{BLASFEO}

\begin{verbatim}
struct d_strmat 
	{
	int bs;
	int m;
	int n;
	int pm;
	int cn;
	double *pA;
	double *dA;
	int use_dA;
	int memory_size;
	};
\end{verbatim}
where the struct members are
\begin{description}
\item[bs] height of the panel
\item[m] number of rows
\item[n] number of columns
\item[pm] number of rows of the matrix as allocated in memory, used for memory alignment
\item[cn] number of rows of the matrix as allocated in memory, used for memory alignment
\item[pA] pointer to a pm$\times$pn array of doubles, the first element is aligned to cache line size
\item[dA] pointer to a min(m,n) array of doubles, used e.g. to store the inverse of the diagonal of the matrix
\item[use\_dA] flag to tell if dA contains useful information
\item[memory\_size] size of the memory (in bytes) needed for pA and pD
\end{description}


\subsection{BLAS}

\begin{verbatim}
struct d_strmat 
	{
	int m; // rows
	int n; // cols
	double *pA; // pointer to a m*n array of doubles
	int memory_size; // size of needed memory
	};
\end{verbatim}
\begin{description}
\item[m] number of rows
\item[n] number of columns
\item[pA] pointer to a m$\times$n array of doubles
\item[memory\_size] size of the memory (in bytes) needed for pA
\end{description}



\section{{\tt strmat} management}

\begin{verbatim}
void d_allocate_strmat(int m, int n, struct d_strmat *sA);
\end{verbatim}

\begin{verbatim}
void d_free_strmat(struct d_strmat *sA);
\end{verbatim}

\begin{verbatim}
int d_size_strmat(int m, int n);
\end{verbatim}

\begin{verbatim}
void d_create_strmat(int m, int n, struct d_strmat *sA, void *memory);
\end{verbatim}



\section{{\tt strmat} conversion}

\begin{verbatim}
void d_cvt_mat2strmat(int m, int n, double *A, int lda, struct d_strmat *sA, 
     int ai, int aj);
\end{verbatim}

\begin{verbatim}
void d_cvt_tran_mat2strmat(int m, int n, double *A, int lda, struct d_strmat *sA, 
     int ai, int aj);
\end{verbatim}

\begin{verbatim}
void d_cvt_strmat2mat(int m, int n, struct d_strmat *sA, int ai, int aj, 
     double *A, int lda);
\end{verbatim}

\begin{verbatim}
void d_cvt_tran_strmat2mat(int m, int n, struct d_strmat *sA, int ai, int aj, 
     double *A, int lda);
\end{verbatim}



\section{{\tt strmat} print}

\begin{verbatim}
void d_print_strmat(int m, int n, struct d_strmat *sA, int ai, int aj);
\end{verbatim}



\end{document}
